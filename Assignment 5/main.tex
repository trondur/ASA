\documentclass{article}
\usepackage[margin=0.5cm]{geometry}
\usepackage[utf8]{inputenc}
\usepackage{graphicx}
\usepackage{listings}
\usepackage{color}
\usepackage{placeins}
\usepackage{hyperref}
\usepackage{xcolor}
\usepackage{caption}
\DeclareCaptionFont{white}{\color{white}}
\DeclareCaptionFormat{listing}{%
  \parbox{\textwidth}{\colorbox{gray}{\parbox{\textwidth}{#1#2#3}}\vskip-4pt}}
\captionsetup[lstlisting]{format=listing,labelfont=white,textfont=white}
\lstset{frame=lrb,xleftmargin=\fboxsep,xrightmargin=-\fboxsep}

\begin{document}
\include{frontpage}
\section*{1:}
Prove that the following problem is undecidable - \emph{Is boolean variable ‘b’ always true in program P?}

\noindent Assuming we have a program \texttt{bIsTrueInP(P)} that takes another program P as an input and decides whether or not the variable \texttt{b} is always true, we can construct the program R in Listing 1.
\belowcaptionskip=-10pt
\begin{lstlisting}[caption=R.cs, language=C]
bool b = false;
P // insert the code of P here.
return b;
\end{lstlisting}
If we run \texttt{bIsTrueInP} with R as an input
\begin{lstlisting}[language=C]
res = bIsTrueInP(R)
\end{lstlisting}
we can conclude that if \texttt{res} is true then the program R halts. If not, program R loops. This way we have solved the halting problem, which has been proven to be undecidable. We have therefore arrived at an contradiction. We must therefore conclude that the problem - \emph{Is boolean variable ‘b’ always true in program P?} - is undecidable

\section*{2:}
\includegraphics[scale = 0.8]{graph2.png}

\section*{3:}
\includegraphics[scale = 0.8]{graph3.png}

\end{document}

