\documentclass[a1paper]{paper}
\usepackage{../mathpartir}
\usepackage[margin=0.5cm]{geometry}

\newcommand{\rskipns}{\ensuremath{\mathsf{[skip_{ns}]}}}
\newcommand{\rassns}{\ensuremath{\mathsf{[ass_{ns}]}}}
\newcommand{\rifttns}{\ensuremath{\mathsf{[if^{tt}_{ns}]}}}
\newcommand{\rifffns}{\ensuremath{\mathsf{[if^{ff}_{ns}]}}}
\newcommand{\rwhilettns}{\ensuremath{\mathsf{[while^{tt}_{ns}]}}}
\newcommand{\rwhileffns}{\ensuremath{\mathsf{[while^{ff}_{ns}]}}}
\newcommand{\rcompns}{\ensuremath{\mathsf{[comp_{ns}]}}}
\newcommand{\rrepeatttns}{\ensuremath{\mathsf{[repeat^{tt}_{ns}]}}}
\newcommand{\rrepeatffns}{\ensuremath{\mathsf{[repeat^{ff}_{ns}]}}}
\newcommand{\rforttns}{\ensuremath{\mathsf{[for^{tt}_{ns}]}}}
\newcommand{\rforffns}{\ensuremath{\mathsf{[for^{ff}_{ns}]}}}

\newcommand{\rskipsos}{\ensuremath{\mathsf{[skip_{sos}]}}}
\newcommand{\rasssos}{\ensuremath{\mathsf{[ass_{sos}]}}}
\newcommand{\rifttsos}{\ensuremath{\mathsf{[if^{tt}_{sos}]}}}
\newcommand{\rifffsos}{\ensuremath{\mathsf{[if^{ff}_{sos}]}}}
\newcommand{\rwhilesos}{\ensuremath{\mathsf{[while_{sos}]}}}
\newcommand{\rcompasos}{\ensuremath{\mathsf{[comp^1_{sos}]}}}
\newcommand{\rcompbsos}{\ensuremath{\mathsf{[comp^2_{sos}]}}}

\newcommand{\rskipas}{\ensuremath{\mathsf{[skip_{as}]}}}
\newcommand{\rassas}{\ensuremath{\mathsf{[ass_{as}]}}}
\newcommand{\rifttas}{\ensuremath{\mathsf{[if^{tt}_{as}]}}}
\newcommand{\rifffas}{\ensuremath{\mathsf{[if^{ff}_{as}]}}}
\newcommand{\rwhileas}{\ensuremath{\mathsf{[while_{as}]}}}
\newcommand{\rcompaas}{\ensuremath{\mathsf{[comp^1_{as}]}}}
\newcommand{\rcompbas}{\ensuremath{\mathsf{[comp^2_{as}]}}}

\newcommand{\cmd}[1]{\ensuremath{\mathsf{\mathbf{{#1}}}}}
\newcommand{\cvar}[1]{\ensuremath{\mathtt{{#1}}}}
\newcommand{\cskip}{\ensuremath{\cmd{skip}}}
\newcommand{\ccomp}[2]{\ensuremath{{#1};\; {#2}}}
\newcommand{\cass}[2]{\ensuremath{\cvar{#1} := {#2}}}
\newcommand{\cif}[3]{\ensuremath{\cmd{if}\; {#1}\; \cmd{then}\; {#2}\; \cmd{else}\; {#3}}}
\newcommand{\cwhile}[2]{\ensuremath{\cmd{while}\; {#1}\; \cmd{do}\; {#2}}}
\newcommand{\crepeat}[2]{\ensuremath{\cmd{repeat}\; {#1}\; \cmd{until}\; {#2}}}
\newcommand{\cfor}[4]{\ensuremath{\cmd{for}\; {\cvar {#1}} := {#2}\; \cmd{to}\; {#3}\; \cmd {do}\; {#4}}}
\newcommand{\ns}[3]{\ensuremath{\langle{#1},\; {#2}\rangle\rightarrow{#3}}}
\newcommand{\sos}[2]{\ensuremath{\langle{#1},\; {#2}\rangle}}
\newcommand{\dev}{\ensuremath{\Rightarrow}}
\newcommand{\devstar}{\ensuremath{\Rightarrow^*}}
\newcommand{\devk}[1]{\ensuremath{\Rightarrow^{#1}}}
\newcommand{\beval}[2]{\ensuremath{\mathcal{B}[{#1}]_{#2}}}
\newcommand{\aeval}[2]{\ensuremath{\mathcal{A}[{#1}]_{#2}}}
\newcommand{\ltrue}{\ensuremath{\mathbf{tt}}}
\newcommand{\lfalse}{\ensuremath{\mathbf{ff}}}
\newcommand{\pand}{\ensuremath{\wedge}}

\newcommand{\cond}[1]{\ensuremath{\{{#1}\}}}
\newcommand{\triple}[3]{\ensuremath{\cond{#1} {#2} \cond{#3}}}

\begin{document}

\section*{2.3:}
\ccomp {\cass z 0} {\cwhile{\cvar{y}\le\cvar{x}} ({\ccomp {\cass z {\cvar z + 1}} {\cass x {\cvar x - \cvar y}}})} \\
\[
\begin{array}[t]{l@{\quad\quad}l}
\begin{array}{r@{\;=\;}l}
s_0 & [x\mapsto 17 y\mapsto 5] \\
s_1 & [x\mapsto 17,\; y\mapsto 5; y\mapsto 0] \\
s_2 & [x\mapsto 17,\; y\mapsto 5; y\mapsto 1] \\
s_3 & [x\mapsto 12,\; y\mapsto 5; y\mapsto 1] \\
s_4 & [x\mapsto 12,\; y\mapsto 5; y\mapsto 2] \\
s_5 & [x\mapsto 7, \; y\mapsto 5; y\mapsto 2] \\
s_6 & [x\mapsto 7, \; y\mapsto 5; y\mapsto 3] \\
s_7 & [x\mapsto 2, \; y\mapsto 5; y\mapsto 3] \\
\end{array}
&
\begin{array}{r@{\quad=\quad}l}
b & \cvar{y}\le\cvar{x} \\
A & \ccomp {\cass z 0} B \\
B & \cwhile b C \\
C & \ccomp {\cass z {\cvar z + 1}} D \\
D & \cass x {\cvar x - \cvar y} \\
\end{array}
\end{array}
\]

\[
\inferrule*[left=\rcompns, width=\textwidth]
{ 
  \inferrule*[left=\rassns]{ }{\ns {\cass z 0} {s} {s_1}}\\
  \inferrule*[left=\rwhilettns, width=\textwidth, right=\ensuremath{\beval {b} {s_1}=\ltrue}]
  {
    \inferrule*[left=\rcompns]
    { 
      \inferrule*[left=\rassns]{ }{\ns {\cass z {\cvar z + 1}} {s_1} {s_2}} \\
      \inferrule*[left=\rassns]{ }{\ns {\cass x {\cvar x - y}} {s_2} {s_3}}
    }
    {\ns C {s_1} {s_3}} \\
    {\inferrule*[left=\rwhilettns, width=\textwidth, right=\ensuremath{\beval {b} {s_3}=\ltrue}]
      {
        \inferrule*[left=\rcompns]
        { 
          \inferrule*[left=\rassns]{ }{\ns {\cass z {\cvar z + 1}} {s_3} {s_4}} \\
          \inferrule*[left=\rassns]{ }{\ns {\cass x {\cvar x - y}} {s_4} {s_5}}
        }
        {\ns C {s_3} {s_5}} \\
        {\inferrule*[left=\rwhilettns, width=\textwidth, right=\ensuremath{\beval {b} {s_5}=\ltrue}]
          {
            \inferrule*[left=\rcompns]
            { 
              \inferrule*[left=\rassns]{ }{\ns {\cass z {\cvar z + 1}} {s_5} {s_6}} \\
              \inferrule*[left=\rassns]{ }{\ns {\cass x {\cvar x - y}} {s_6} {s_7}}
            }
            {\ns C {s_5} {s_7}} \\
            {\inferrule*[left=\rwhileffns, right=\ensuremath{\beval {b} {s_7}=\lfalse}]{ }{\ns B {s_7} {s_7}}}
          }
          {\ns B {s_5} {s_7}}
        }
      }
      {\ns B {s_3} {s_7}}
    }
  }
  {\ns B {s_1} {s_7}}    
}
{\ns A {s_0} {s_7}}
\]

\section*{2.4:}
Program one will not terminate, because $\neg(\cvar{x}= 1)$ will never be false when the initial state of x is 0. So the inference tree will expand forever. \\
\cwhile {\neg(\cvar{x}= 1)} (\ccomp {\cass y y*x} {\cass x x-1}) \\
\[
\begin{array}[t]{l@{\quad\quad}l}
\begin{array}{r@{\;=\;}l}
s_0 & [x\mapsto 0, y\mapsto 0] \\
s_1 & [x\mapsto 0, y\mapsto -1] \\
s_2 & [x\mapsto 0, y\mapsto -1] \\
s_3 & [x\mapsto 0, y\mapsto -2] \\
s_4 & [x\mapsto 0, y\mapsto -2] \\
\end{array}
&
\begin{array}{r@{\quad=\quad}l}
b & \neg(\cvar{x}= 1) \\
A & \cwhile b B \\
B & \ccomp {\cass y y*x} C \\
C & \cass x x-1 \\
\end{array}
\end{array}
\]

\[
\inferrule*[left=\rwhilettns, width=\textwidth, right={\beval b {s_0} = \ltrue}]
{
	\inferrule*[left=\rcompns]
	{ 
		\inferrule*[left=\rassns]{ }{\ns {\cass y {\cvar y * \cvar x}} {s_0} {s_1}} \\
      	\inferrule*[left=\rassns]{ }{\ns {\cass x {\cvar x - 1}} {s_1} {s_2}}
	}
	{\ns B {s_0} {s_2}}
	\inferrule*[left=\rwhilettns, width=\textwidth, right={\beval b {s_2} = \ltrue}]
	{
		\inferrule*[left=\rcompns]
		{ 
			\inferrule*[left=\rassns]{ }{\ns {\cass y {\cvar y * \cvar x}} {s_2} {s_3}} \\
	      	\inferrule*[left=\rassns]{ }{\ns {\cass x {\cvar x - 1}} {s_3} {s_4}}
		}
		{\ns B {s_2} {s_4}}{\ns {\cwhile b B} {s_4}{s_x}}
	}
	{\ns A {s_2} {s_x}}
}
{\ns A {s_0} {s_x}}
\]
Program three will never terminate because \textbf{true} will never be false, so the inference tree will expand forever. \\
\cwhile {1\le \cvar{x}} (\ccomp {\cass y y*x} {\cass x x-1}) \\
\[
\begin{array}[t]{l@{\quad\quad}l}
\begin{array}{r@{\;=\;}l}
s_0 & [x\mapsto 0, y\mapsto 0] \\
s_1 & [x\mapsto 0, y\mapsto 0] \\
s_2 & [x\mapsto -1, y\mapsto 0] \\
\end{array}
&
\begin{array}{r@{\quad=\quad}l}
b & 1\le x \\
A & \cwhile b B \\
B & \ccomp {\cass y {y*x}} {C} \\
C & \cass x {x-1}
\end{array}
\end{array}
\]
\[
\inferrule*[left=\rwhilettns, width=\textwidth, right={\beval b {s_0} = \ltrue}]
{
	\inferrule*[left=\rcompns]
	{ 
		\inferrule*[left=\rassns]{ }{\ns {\cass y {\cvar y * \cvar x}} {s_0} {s_1}} \\
      	\inferrule*[left=\rassns]{ }{\ns {\cass x {\cvar x - 1}} {s_1} {s_2}}
	}
	{\ns B {s_0} {s_2}}{\ns {\cwhile b B} {s_2}{s_x}}
}
{\ns A {s_0} {s_x}}
\]



\noindent \cwhile {true}  \cskip
\[
\begin{array}[t]{l@{\quad\quad}l}
\begin{array}{r@{\;=\;}l}
s_0 & [] \\
\end{array}
&
\begin{array}{r@{\quad=\quad}l}
b & \ltrue \\
A & \cwhile b \cskip \\
\end{array}
\end{array}
\]
\[
\inferrule*[left=\rwhilettns, width=\textwidth, right={\beval b {s_0} = \ltrue}]
{
	{\ns \cskip {s_0} {s_0}},\\{\ns {\cwhile b \cskip} {s_0} {s_x}}
}
{\ns A {s_0} {s_x}}
\]

\section*{2.6:}
Show that $S1;(S2;S3)$ and $(S1;S2);S3$ are semantically equivalent: \\
\[
	\begin{array}{r@{\quad=\quad}l}
	C_1 & \ccomp {S_1}{C_2} \\
	C_2 & \ccomp {S_2}{S_3} \\
	C_3 & \ccomp {C_4}{S_3} \\
	C_4 & \ccomp {S_1}{S_2} \\

	\end{array}
\]
\[
	\inferrule*[left=\rcompns]
	{ 
		{\ns {S_1} {s_0} {s_1}} \\
	    \inferrule*[left=\rcompns]
		{
			{\ns {S_2} {s_1} {s_2}} \\
			{\ns {S_3} {s_2} {s_3}}
		}{\ns {C_2} {s_1} {s_3}}
	}
	{\ns {\ccomp {S_1} {C_2}} {s_0} {s_3}}
\]

\[
	\inferrule*[left=\rcompns]
	{ 
	    \inferrule*[left=\rcompns]
		{
			{\ns {S_1} {s_0} {s_1}} \\
			{\ns {S_2} {s_1} {s_2}}
		}{\ns {C_4} {s_0} {s_2}}{\ns {S_3} {s_2}{s_3}}
	}
	{\ns {\ccomp {C_3} {S_1}} {s_0} {s_3}}
\]

\noindent Show that $S1;S2$ and $S2;S1$ are not always semantically equivalent: \\

\[
\begin{array}[t]{l@{\quad\quad}l}
\begin{array}{r@{\;=\;}l}
s_0 & [] \\
s_1 & [x\mapsto 3] \\
s_2 & [x\mapsto 5] \\
s_3 & [x\mapsto 5] \\
s_4 & [x\mapsto 3] \\
\end{array}
&
\begin{array}{r@{\quad=\quad}l}
C_1 & \ccomp {S_1}{S_2} \\
C_2 & \ccomp {S_2}{S_1} \\
S_1 & \cass x 3 \\
S_2 & \cass x 5 \\
\end{array}
\end{array}
\]


\[
	\inferrule*[left=\rcompns]
	{ 
		\inferrule*[left=\rassns]{ }{\ns {\cass x 3} {s_0} {s_1}}\\
	    \inferrule*[left=\rassns]{ }{\ns {\cass x 5} {s_1} {s_2}}
	}
	{\ns {C_1} {s_0} {s_2}}
\]
\[
	\inferrule*[left=\rcompns]
	{ 
		\inferrule*[left=\rassns]{ }{\ns {\cass x 5} {s_0} {s_3}}\\
	    \inferrule*[left=\rassns]{ }{\ns {\cass x 3} {s_3} {s_4}}
	}
	{\ns {C_2} {s_0} {s_4}}
\]

\section*{2.7:}
The While language can be extended with the \crepeat S b statement:
\[
\inferrule*[left=\rrepeatffns, right={\beval b {s_0} = \lfalse}]
{
	\cskip
}{\ns {\crepeat S b} {s_0} {s_0}}
\]
\[
\inferrule*[left=\rrepeatttns, right={\beval b {s_1} = \ltrue}]
{
	{\ns S {s_0} {s_1}},\\ 
	{\ns {\crepeat S b} {s_1} {s_x} }
}{\ns {\crepeat S b} {s_0} {s_x}}
\]

\section*{2.8:}
The While language can be extended with the: \cfor A {a_1} {a_2} S, statement:
\[
\inferrule*[left=\rforffns, right={\beval {a_1 = a_2} {s_1} = \lfalse}]
{
	\inferrule*[left=\rassns]{ }{\ns {\cass A {a_0}} {s_0} {s_1}}\\
	{\ns S {s_1} {s_2}} \\
	\inferrule*[left=\rassns]{ }{\ns {\cass A {\cvar a_0 + 1}} {s_2} {s_3}}\\
	{\ns {\cfor A {a_1} {a_2} S} {s_3} {s_4} }
}
{\ns {\cfor A {a_1} {a_2} S} {s_0} {s_1}}
\]
\[
\inferrule*[left=\rforttns, right={\beval {a_1 = a_2} {s_1} = \ltrue}]
{
	\inferrule*[left=\rassns]{ }{\ns {\cass A {a_0}} {s_0} {s_1}}\\ {\ns \cskip {s_1}{s_1}}
}
{\ns {\cfor A {a_1} {a_2} S} {s_0} {s_4}}
\]
\end{document}