\documentclass[a0paper, landscape]{paper}
\usepackage{../mathpartir}
\usepackage[margin=0.5cm]{geometry}

\newcommand{\rskipns}{\ensuremath{\mathsf{[skip_{ns}]}}}
\newcommand{\rassns}{\ensuremath{\mathsf{[ass_{ns}]}}}
\newcommand{\rifttns}{\ensuremath{\mathsf{[if^{tt}_{ns}]}}}
\newcommand{\rifffns}{\ensuremath{\mathsf{[if^{ff}_{ns}]}}}
\newcommand{\rwhilettns}{\ensuremath{\mathsf{[while^{tt}_{ns}]}}}
\newcommand{\rwhileffns}{\ensuremath{\mathsf{[while^{ff}_{ns}]}}}
\newcommand{\rcompns}{\ensuremath{\mathsf{[comp_{ns}]}}}
\newcommand{\rrepeatttns}{\ensuremath{\mathsf{[repeat^{tt}_{ns}]}}}
\newcommand{\rrepeatffns}{\ensuremath{\mathsf{[repeat^{ff}_{ns}]}}}
\newcommand{\rforttns}{\ensuremath{\mathsf{[for^{tt}_{ns}]}}}
\newcommand{\rforffns}{\ensuremath{\mathsf{[for^{ff}_{ns}]}}}

\newcommand{\rskipsos}{\ensuremath{\mathsf{[skip_{sos}]}}}
\newcommand{\rasssos}{\ensuremath{\mathsf{[ass_{sos}]}}}
\newcommand{\rifttsos}{\ensuremath{\mathsf{[if^{tt}_{sos}]}}}
\newcommand{\rifffsos}{\ensuremath{\mathsf{[if^{ff}_{sos}]}}}
\newcommand{\rwhilesos}{\ensuremath{\mathsf{[while_{sos}]}}}
\newcommand{\rcompasos}{\ensuremath{\mathsf{[comp^1_{sos}]}}}
\newcommand{\rcompbsos}{\ensuremath{\mathsf{[comp^2_{sos}]}}}

\newcommand{\rskipas}{\ensuremath{\mathsf{[skip_{as}]}}}
\newcommand{\rassas}{\ensuremath{\mathsf{[ass_{as}]}}}
\newcommand{\rifttas}{\ensuremath{\mathsf{[if^{tt}_{as}]}}}
\newcommand{\rifffas}{\ensuremath{\mathsf{[if^{ff}_{as}]}}}
\newcommand{\rwhileas}{\ensuremath{\mathsf{[while_{as}]}}}
\newcommand{\rcompaas}{\ensuremath{\mathsf{[comp^1_{as}]}}}
\newcommand{\rcompbas}{\ensuremath{\mathsf{[comp^2_{as}]}}}

\newcommand{\cmd}[1]{\ensuremath{\mathsf{\mathbf{{#1}}}}}
\newcommand{\cvar}[1]{\ensuremath{\mathtt{{#1}}}}
\newcommand{\cskip}{\ensuremath{\cmd{skip}}}
\newcommand{\ccomp}[2]{\ensuremath{{#1};\; {#2}}}
\newcommand{\cass}[2]{\ensuremath{\cvar{#1} := {#2}}}
\newcommand{\cif}[3]{\ensuremath{\cmd{if}\; {#1}\; \cmd{then}\; {#2}\; \cmd{else}\; {#3}}}
\newcommand{\cwhile}[2]{\ensuremath{\cmd{while}\; {#1}\; \cmd{do}\; {#2}}}
\newcommand{\crepeat}[2]{\ensuremath{\cmd{repeat}\; {#1}\; \cmd{until}\; {#2}}}
\newcommand{\cfor}[4]{\ensuremath{\cmd{for}\; {\cvar {#1}} := {#2}\; \cmd{to}\; {#3}\; \cmd {do}\; {#4}}}
\newcommand{\ns}[3]{\ensuremath{\langle{#1},\; {#2}\rangle\rightarrow{#3}}}
\newcommand{\sos}[2]{\ensuremath{\langle{#1},\; {#2}\rangle}}
\newcommand{\dev}{\ensuremath{\Rightarrow}}
\newcommand{\devstar}{\ensuremath{\Rightarrow^*}}
\newcommand{\devk}[1]{\ensuremath{\Rightarrow^{#1}}}
\newcommand{\beval}[2]{\ensuremath{\mathcal{B}[{#1}]_{#2}}}
\newcommand{\aeval}[2]{\ensuremath{\mathcal{A}[{#1}]_{#2}}}
\newcommand{\ltrue}{\ensuremath{\mathbf{tt}}}
\newcommand{\lfalse}{\ensuremath{\mathbf{ff}}}
\newcommand{\pand}{\ensuremath{\wedge}}

\newcommand{\cond}[1]{\ensuremath{\{{#1}\}}}
\newcommand{\triple}[3]{\ensuremath{\cond{#1} {#2} \cond{#3}}}

\begin{document}

\section*{2.3:}

\[
\inferrule*[left=\rcompns, width=\textwidth]{ 
  \inferrule*[left=\rassns]{ }{\ns {\cass x 3} {[]} {[x\mapsto 3]}}\\
  \inferrule*[left=\rcompns]{ 
    \inferrule*[left=\rassns]{ }{\ns {\cass y 0} {[x\mapsto 3]} {[x\mapsto 3,\; y\mapsto 0]}}\\
    \inferrule*[left=\rwhilettns, width=\textwidth, right=\ensuremath{\beval {\neg(\cvar{x}\le 0)} {[x\mapsto 3,\; y\mapsto 0]}=\ltrue}]{
      \inferrule*[left=\rcompns]{ 
        \inferrule*[left=\rassns]{ }{\ns {\cass y {\cvar y + \cvar x}} {[x\mapsto 3,\; y\mapsto 0]} {[x\mapsto 3,\; y\mapsto 3]}} \\
        \inferrule*[left=\rassns]{ }{\ns {\cass x {\cvar x - 1}} {[x\mapsto 3,\; y\mapsto 3]} {[x\mapsto 2,\; y\mapsto 3]}}
      }{\ns D {[x\mapsto 3,\; y\mapsto 0]} {[x\mapsto 2,\; y\mapsto 3]}} \\
       {\inferrule*[left=\rwhilettns, width=\textwidth, right=\ensuremath{\beval {\neg(\cvar{x}\le 0)} {[x\mapsto 2,\; y\mapsto 3]}=\ltrue}]{
           \inferrule*[left=\rcompns]{ 
             \inferrule*[left=\rassns]{ }{\ns {\cass y {\cvar y + \cvar x}} {[x\mapsto 2,\; y\mapsto 3]} {[x\mapsto 2,\; y\mapsto 5]}} \\
             \inferrule*[left=\rassns]{ }{\ns {\cass x {\cvar x - 1}} {[x\mapsto 2,\; y\mapsto 5]} {[x\mapsto 1,\; y\mapsto 5]}}
           }{\ns D {[x\mapsto 2,\; y\mapsto 3]} {[x\mapsto 1,\; y\mapsto 5]}} \\
                      {\inferrule*[left=\ensuremath{T_3\quad=\quad}\rwhilettns, width=\textwidth, right=\ensuremath{\beval {\neg(\cvar{x}\le 0)} {[x\mapsto 1,\; y\mapsto 5]}=\ltrue}]{
  \inferrule*[left=\rcompns]{ 
    \inferrule*[left=\rassns]{ }{\ns {\cass y {\cvar y + \cvar x}} {[x\mapsto 1,\; y\mapsto 5]} {[x\mapsto 1,\; y\mapsto 6]}} \\
    \inferrule*[left=\rassns]{ }{\ns {\cass x {\cvar x - 1}} {[x\mapsto 1,\; y\mapsto 6]} {[x\mapsto 0,\; y\mapsto 6]}}
  }{\ns D {[x\mapsto 1,\; y\mapsto 5]} {[x\mapsto 0,\; y\mapsto 6]}} \\
  {\inferrule*[left=\rwhileffns, right=\ensuremath{\beval {\neg(\cvar{x}\le 0)} {[x\mapsto 0,\; y\mapsto 6]}=\lfalse}]{ }{\ns C {[x\mapsto 0,\; y\mapsto 6]} {[x\mapsto 0,\; y\mapsto 6]}}}
}{\ns C {[x\mapsto 1,\; y\mapsto 5]} {[x\mapsto 0,\; y\mapsto 6]}}
}
         }{\ns C {[x\mapsto 2,\; y\mapsto 3]} {[x\mapsto 0,\; y\mapsto 6]}}
       }
    }{\ns C {[x\mapsto 3,\; y\mapsto 0]} {[x\mapsto 0,\; y\mapsto 6]}}    
  }{\ns B {[x\mapsto 3]} {[x\mapsto 0,\; y\mapsto 6]}}
}{\ns A {[]} {[x\mapsto 0,\; y\mapsto 6]}}
\]

\section*{2.4:}
Do the following statements terminate:\\
\cwhile {\neg(\cvar{x}= 1)} (\ccomp {\cass y y*x} {\cass x x-1}) \\

\cwhile {1\le \cvar{x}} (\ccomp {\cass y y*x} {\cass x x-1}) \\

\cwhile {true}  \cskip \\

\section*{2.6:}
Show that $S1;(S2;S3)$ and $(S1;S2);S3$ are semantically equivalent: \\

Show that $S1;S2$ and $S2;S1$ are not always semantically equivalent:



\section*{2.7:}
While can be extended with the:
 \crepeat S b statement:

\section*{2.8:}

\end{document}