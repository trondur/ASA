\documentclass[a4paper]{paper}
\usepackage{../mathpartir}
\usepackage[margin=0.5cm]{geometry}

\newcommand{\rskipns}{\ensuremath{\mathsf{[skip_{ns}]}}}
\newcommand{\rassns}{\ensuremath{\mathsf{[ass_{ns}]}}}
\newcommand{\rifttns}{\ensuremath{\mathsf{[if^{tt}_{ns}]}}}
\newcommand{\rifffns}{\ensuremath{\mathsf{[if^{ff}_{ns}]}}}
\newcommand{\rwhilettns}{\ensuremath{\mathsf{[while^{tt}_{ns}]}}}
\newcommand{\rwhileffns}{\ensuremath{\mathsf{[while^{ff}_{ns}]}}}
\newcommand{\rcompns}{\ensuremath{\mathsf{[comp_{ns}]}}}
\newcommand{\rrepeatttns}{\ensuremath{\mathsf{[repeat^{tt}_{ns}]}}}
\newcommand{\rrepeatffns}{\ensuremath{\mathsf{[repeat^{ff}_{ns}]}}}
\newcommand{\rforttns}{\ensuremath{\mathsf{[for^{tt}_{ns}]}}}
\newcommand{\rforffns}{\ensuremath{\mathsf{[for^{ff}_{ns}]}}}

\newcommand{\rskipsos}{\ensuremath{\mathsf{[skip_{sos}]}}}
\newcommand{\rasssos}{\ensuremath{\mathsf{[ass_{sos}]}}}
\newcommand{\rifttsos}{\ensuremath{\mathsf{[if^{tt}_{sos}]}}}
\newcommand{\rifffsos}{\ensuremath{\mathsf{[if^{ff}_{sos}]}}}
\newcommand{\rwhilesos}{\ensuremath{\mathsf{[while_{sos}]}}}
\newcommand{\rcompasos}{\ensuremath{\mathsf{[comp^1_{sos}]}}}
\newcommand{\rcompbsos}{\ensuremath{\mathsf{[comp^2_{sos}]}}}

\newcommand{\rskipas}{\ensuremath{\mathsf{[skip_{as}]}}}
\newcommand{\rassas}{\ensuremath{\mathsf{[ass_{as}]}}}
\newcommand{\rifttas}{\ensuremath{\mathsf{[if^{tt}_{as}]}}}
\newcommand{\rifffas}{\ensuremath{\mathsf{[if^{ff}_{as}]}}}
\newcommand{\rwhileas}{\ensuremath{\mathsf{[while_{as}]}}}
\newcommand{\rcompaas}{\ensuremath{\mathsf{[comp^1_{as}]}}}
\newcommand{\rcompbas}{\ensuremath{\mathsf{[comp^2_{as}]}}}

\newcommand{\cmd}[1]{\ensuremath{\mathsf{\mathbf{{#1}}}}}
\newcommand{\cvar}[1]{\ensuremath{\mathtt{{#1}}}}
\newcommand{\cskip}{\ensuremath{\cmd{skip}}}
\newcommand{\ccomp}[2]{\ensuremath{{#1};\; {#2}}}
\newcommand{\cass}[2]{\ensuremath{\cvar{#1} := {#2}}}
\newcommand{\cif}[3]{\ensuremath{\cmd{if}\; {#1}\; \cmd{then}\; {#2}\; \cmd{else}\; {#3}}}
\newcommand{\cwhile}[2]{\ensuremath{\cmd{while}\; {#1}\; \cmd{do}\; {#2}}}
\newcommand{\crepeat}[2]{\ensuremath{\cmd{repeat}\; {#1}\; \cmd{until}\; {#2}}}
\newcommand{\cfor}[4]{\ensuremath{\cmd{for}\; {\cvar {#1}} := {#2}\; \cmd{to}\; {#3}\; \cmd {do}\; {#4}}}
\newcommand{\ns}[3]{\ensuremath{\langle{#1},\; {#2}\rangle\rightarrow{#3}}}
\newcommand{\sos}[2]{\ensuremath{\langle{#1},\; {#2}\rangle}}
\newcommand{\dev}{\ensuremath{\Rightarrow}}
\newcommand{\devstar}{\ensuremath{\Rightarrow^*}}
\newcommand{\devk}[1]{\ensuremath{\Rightarrow^{#1}}}
\newcommand{\beval}[2]{\ensuremath{\mathcal{B}[{#1}]_{#2}}}
\newcommand{\aeval}[2]{\ensuremath{\mathcal{A}[{#1}]_{#2}}}
\newcommand{\ltrue}{\ensuremath{\mathbf{tt}}}
\newcommand{\lfalse}{\ensuremath{\mathbf{ff}}}
\newcommand{\pand}{\ensuremath{\wedge}}

\newcommand{\cond}[1]{\ensuremath{\{{#1}\}}}
\newcommand{\triple}[3]{\ensuremath{\cond{#1} {#2} \cond{#3}}}

\begin{document}

\section*{Exercise 1}

\subsection*{a)} The data type Strange is strange, because it lacks a base case. Therefore derivation sequences will be infinite. \\

\subsection*{b)} Provide an induction principle for Strange.\\
$Strange :=\\
| C1\;bool\;Strange\\
| C2\;int\;Strange\\
$
\[
\inferrule*[width=\textwidth]
{
	\forall (x\;:\;Strange).\;P\;x \rightarrow(int|bool,\;Strange)
}{P\;Strange}
\]
\subsection*{c)} Assume that you have an element of type Strange. Now prove False.


\section*{Exercise 2}


\hspace{0.5cm}$\rasssos \forall a\;s.\;P(\cass x a)\;s\;(s[x \rightarrow \aeval a s])$\\

$\rskipsos \forall s.\;P\;\cskip\;s\;s$\\

$\rcompasos$\\

$\rcompbsos$\\

$\rifttsos \forall s\;s'\;b\;{S_1}\;{S_2}.\;\beval b s = \ltrue \dev {\ns {S_1} s {s'}} \dev P\;{S_1}\;s\;s' \dev P \ns {\cif b {S_1} {S_2}} s s'$\\

$\rifffsos \forall s\;s'\;b\;{S_1}\;{S_2}.\;\beval b s = \lfalse \dev {\ns {S_2} s {s'}} \dev P\;{S_2}\;s\;s' \dev P \ns {\cif b {S_1} {S_2}} s s'$\\

$\rwhilesos \forall s\;s'\;s''\;b\;S.\;{\ns S s {s''}} \dev { \ns{\cwhile b S} {s''} s'}\\
P\;S\;s\;{s''} \dev P \ns {\cwhile b S} {s''} {s'} \dev P\;(\cwhile b S)\;{s''}\;{s'} \dev \beval b s = \ltrue \dev P\;(\cwhile b S)\;s\;s$\\

$\rwhilesos \forall b\;S.\;\beval b s = \lfalse \dev P\;(\cwhile b S)\;s\;s
$\\

\section*{Exercise 3}
Prove that if $(S_1, s) \devk k s'$ then $(S_1;S_2, s) \devk k (S_2, s')$

\noindent We can assume A:

$$({S_1},s) \devk k {s'}$$
Only two derivation rules can be used to derive B:

$$({S_1};{S_2}, s)\devk k ({S_2}, s')$$
The first is C:
\[
\inferrule*[left=\rcompasos, width=\textwidth]
{
	{({S_1}, s) \dev ({S'_1}, s')}
}
{({S_1};{S_2}, s)\dev ({S'_1};{S_2}, s')}
\]
and the second D:
\[
\inferrule*[left=\rcompbsos, width=\textwidth]
{
	{({S_1}, s) \dev {s'}}
}
{({S_1};{S_2}, s)\dev ({{S_2}, s')}}
\]

\noindent Since we assume A, that $S_1$ terminates in $s'$ in k steps, we have the assumptions needed to use D and therefore show that the execution of $S_1$ is not influence by the following statements.

\section*{Exercise 4}

\end{document}